\documentclass[11pt]{article}

% This file will be kept up-to-date at the following GitHub repository:
%
% https://github.com/automl-conf/LatexTemplate
%
% Please file any issues/bug reports, etc. you may have at:
%
% https://github.com/automl-conf/LatexTemplate/issues

\usepackage{microtype} % microtypography
\usepackage{booktabs}  % tables
\usepackage{url}  % urls

% AMS math
\usepackage{amsmath}
\usepackage{amsthm}

% With no package options, the submission will be anonymized, the supplemental
% material will be suppressed, and line numbers will be added to the manuscript.
%
% To compile a non-anonymized camera-ready version, add the [final] option:
%
% \usepackage[final]{automl}
%
% To compile for the workshop track, add the [workshop] option:
%
% \usepackage[workshop]{automl}
% \usepackage[final,workshop]{automl}
%
% To compile for the ABCD track, add the [abcd] option:
%
% \usepackage[abcdtrack]{automl}
% \usepackage[final,abcdtrack]{automl}
%
%
% To compile a pre-print version that can be uploaded to arXiv, add a [preprint] option:
%
% \usepackage[preprint]{automl}
%
% The [hidesupplement] option may be used to hide the supplementary material from any of the
% above commands, e.g.:
%
% \usepackage[hidesupplement,final,workshop]{automl}
%

\usepackage[hidesupplement]{automl}

% You may use any reference style as long as you are consistent throughout the
% document. As a default we suggest author--year citations; for bibtex and
% natbib you may use:

\usepackage{natbib}
\bibliographystyle{apalike}

% and for biber and biblatex you may use:

% \usepackage[%
%   backend=biber,
%   style=authoryear-comp,
%   sortcites=true,
%   natbib=true,
%   giveninits=true,
%   maxcitenames=2,
%   doi=false,
%   url=true,
%   isbn=false,
%   dashed=false
% ]{biblatex}
% \addbibresource{...}

\title{Example Submission for AutoML 2023}

% The syntax for adding an author is
%
% \author[i]{\nameemail{author name}{author email}}
%
% where i is an affiliation counter. Authors may have
% multiple affiliations; e.g.:
%
% \author[1,2]{\nameemail{Anonymous}{anonymous@example.com}}

\author[1]{\nameemail{Author 1}{email1@example.com}}

% the list might continue:
% \author[2,3]{\nameemail{Author 2}{email2@example.com}}
% \author[3]{\nameemail{Author 3}{email3@example.com}}
% \author[4]{\nameemail{Author 4}{email4@example.com}}

% if you need to force a linebreak in the author list, prepend an \author entry
% with \\:

% \author[3]{\\\nameemail{Author 5}{email5@example.com}}

% Specify corresponding affiliations after authors, referring to counter used in
% \author:

\affil[1]{Institution 1}

% the list might continue:
% \affil[2]{Institution 2}
% \affil[3]{Institution 3}
% \affil[4]{Institution 4}

% define PDF metadata, please fill in to aid in accessibility of the resulting PDF
\hypersetup{%
  pdfauthor={}, % will be reset to "Anonymous" unless the "final" package option is given
  pdftitle={},
  pdfsubject={},
  pdfkeywords={}
}

\begin{document}

\maketitle

\begin{abstract}
\end{abstract}

\section{Broader Impact Statement}

% The 9 pages allocated for the main paper must include a broader impact
% statement regarding the approach, datasets and applications proposed/used in
% your paper. It should reflect on the environmental, ethical and societal
% implications of your work. The statement should require at most one page and
% must be included both at submission and camera-ready time.
%
% If authors have reflected on their work and determined that there are no
% likely negative broader impacts, they may use the following statement:
%
% After careful reflection, the authors have determined that this work presents
% no notable negative impacts to society or the environment.
%
% This section is included in the template as a default, but you can also place these
% discussions anywhere else in the main paper, e.g., in the introduction/future work.
%
% The Centre for the Governance of AI has written an excellent guide for writing
% good broader impact statements (for the NeurIPS conference) that may be a
% useful resource for AutoML-Conf authors:
%
% https://medium.com/@GovAI/a-guide-to-writing-the-neurips-impact-statement-4293b723f832

\section{Submission Checklist}

% The submission checklist is a combination of the NeurIPS '21 checklist:
%
%   https://neurips.cc/Conferences/2021/PaperInformation/PaperChecklist
%
% and the NAS checklist:
%
%   https://www.automl.org/wp-content/uploads/NAS/NAS_checklist.pdf
%
% For each question, change the default \answerTODO{} to either:
%
%     \answerYes{[justification]},
%     \answerNo{[justification]}, or
%     \answerNA{[justification]}.
%
% *You must include a brief justification to your answer,* either by
% referencing the appropriate section of your paper or providing a brief inline
% description.  For example:
%
% - Did you include the license of the code and datasets?
%   \answerYes{See Section~\ref{sec:code}.}
%
% - Did you include all the code for running experiments?
%   \answerNo{We include the code we wrote, but it depends on proprietary
%   libraries for executing on a compute cluster and as such will not be
%   runnable without modifications. We also include a runnable sequential
%   version of the code that we also report experiments in the paper with.}
%
% - Did you include the license of the datasets?
%   \answerNA{Our experiments were conducted on publicly available datasets and
%   we did not introduce new datasets.}
%
% Please note that if you answer a question with \answerNo{}, we expect that you
% compensate for it (e.g., if you cannot provide the full evaluation code, you
% should at least provide code for a minimal reproduction of the main insights
% of your paper).
%
% Please do not modify the questions and only use the provided macros for your
% answers. Note that this section does not count towards the page limit.

\begin{enumerate}
\item For all authors\dots
  %
  \begin{enumerate}
  \item Do the main claims made in the abstract and introduction accurately
    reflect the paper's contributions and scope?
    %
    \answerTODO{}
    %
  \item Did you describe the limitations of your work?
    %
    \answerTODO{}
    %
  \item Did you discuss any potential negative societal impacts of your work?
    %
    \answerTODO{}
    %
  \item Have you read the ethics author's and review guidelines and ensured that
    your paper conforms to them? \url{https://automl.cc/ethics-accessibility/}
    %
    \answerTODO{}
    %
  \end{enumerate}
  %
\item If you are including theoretical results\dots
  %
  \begin{enumerate}
  \item Did you state the full set of assumptions of all theoretical results?
    %
    \answerTODO{}
    %
  \item Did you include complete proofs of all theoretical results?
    %
    \answerTODO{}
    %
  \end{enumerate}
  %
\item If you ran experiments\dots
  %
  \begin{enumerate}
  \item Did you include the code, data, and instructions needed to reproduce the
    main experimental results, including all requirements (e.g.,
    \texttt{requirements.txt} with explicit version), an instructive
    \texttt{README} with installation, and execution commands (either in the
    supplemental material or as a \textsc{url})?
    %
    \answerTODO{}
    %
  \item Did you include the raw results of running the given instructions on the
    given code and data?
    %
    \answerTODO{}
    %
  \item Did you include scripts and commands that can be used to generate the
    figures and tables in your paper based on the raw results of the code, data,
    and instructions given?
    %
    \answerTODO{}
    %
  \item Did you ensure sufficient code quality such that your code can be safely
    executed and the code is properly documented?
    %
    \answerTODO{}
    %
  \item Did you specify all the training details (e.g., data splits,
    pre-processing, search spaces, fixed hyperparameter settings, and how they
    were chosen)?
    %
    \answerTODO{}
    %
  \item Did you ensure that you compared different methods (including your own)
    exactly on the same benchmarks, including the same datasets, search space,
    code for training and hyperparameters for that code?
    %
    \answerTODO{}
    %
  \item Did you run ablation studies to assess the impact of different
    components of your approach?
    %
    \answerTODO{}
    %
  \item Did you use the same evaluation protocol for the methods being compared?
    %
    \answerTODO{}
    %
  \item Did you compare performance over time?
    %
    \answerTODO{}
    %
  \item Did you perform multiple runs of your experiments and report random seeds?
    %
    \answerTODO{}
    %
  \item Did you report error bars (e.g., with respect to the random seed after
    running experiments multiple times)?
    %
    \answerTODO{}
    %
  \item Did you use tabular or surrogate benchmarks for in-depth evaluations?
    %
    \answerTODO{}
    %
  \item Did you include the total amount of compute and the type of resources
    used (e.g., type of \textsc{gpu}s, internal cluster, or cloud provider)?
    %
    \answerTODO{}
    %
  \item Did you report how you tuned hyperparameters, and what time and
    resources this required (if they were not automatically tuned by your AutoML
    method, e.g. in a \textsc{nas} approach; and also hyperparameters of your
    own method)?
    %
    \answerTODO{}
    %
  \end{enumerate}
  %
\item If you are using existing assets (e.g., code, data, models) or
  curating/releasing new assets\dots
  %
  \begin{enumerate}
  \item If your work uses existing assets, did you cite the creators?
    %
    \answerTODO{}
    %
  \item Did you mention the license of the assets?
    %
    \answerTODO{}
    %
  \item Did you include any new assets either in the supplemental material or as
    a \textsc{url}?
    %
    \answerTODO{}
    %
  \item Did you discuss whether and how consent was obtained from people whose
    data you're using/curating?
    %
    \answerTODO{}
    %
  \item Did you discuss whether the data you are using/curating contains
    personally identifiable information or offensive content?
    %
    \answerTODO{}
    %
  \end{enumerate}
  %
\item If you used crowdsourcing or conducted research with human subjects\dots
  %
  \begin{enumerate}
  \item Did you include the full text of instructions given to participants and
    screenshots, if applicable?
    %
    \answerTODO{}
    %
  \item Did you describe any potential participant risks, with links to
    Institutional Review Board (\textsc{irb}) approvals, if applicable?
    %
    \answerTODO{}
    %
  \item Did you include the estimated hourly wage paid to participants and the
    total amount spent on participant compensation?
    %
    \answerTODO{}
    %
  \end{enumerate}
\end{enumerate}

% content will be automatically hidden during submission
\begin{acknowledgements}

\end{acknowledgements}

% print bibliography -- for bibtex / natbib, use:

% \bibliography{...}

% and for biber / biblatex, use:

% \printbibliography

% supplemental material -- everything hereafter will be suppressed during
% submission time if the hidesupplement option is provided!
\appendix

\end{document}
