\documentclass[11pt]{article}

% This file will be kept up-to-date at the following GitHub repository:
%
% https://github.com/automl-conf/LatexTemplate
%
% Please file any issues/bug reports, etc. you may have at:
%
% https://github.com/automl-conf/LatexTemplate/issues

\usepackage{microtype} % microtypography
\usepackage{booktabs}  % tables
\usepackage{url}  % urls
\usepackage{csquotes}

% AMS math
\usepackage{amsmath}
\usepackage{amsthm}

% With no package options, the submission will be anonymized, the supplemental
% material will be suppressed, and line numbers will be added to the manuscript.
%
% To hide the supplementary material (e.g., for the first submission deadline),
% use the [hidesupplement] option:
%
% \usepackage[hidesupplement]{automl}
%
% To compile a non-anonymized camera-ready version, add the [final] option (for
% the main track), or the [finalworkshop] option (for the workshop track), e.g.,
%
% \usepackage[final]{automl}
% \usepackage[finalworkshop]{automl}
%
% or
%
% \usepackage[final, hidesupplement]{automl}
% \usepackage[finalworkshop, hidesupplement]{automl}

\usepackage[final]{automl}

% You may use any reference style as long as you are consistent throughout the
% document. As a default we suggest author--year citations; for bibtex and
% natbib you may use:

\usepackage{natbib}
\bibliographystyle{apalike}

% and for biber and biblatex you may use:

% \usepackage[%
%   backend=biber,
%   style=authoryear-comp,
%   sortcites=true,
%   natbib=true,
%   giveninits=true,
%   maxcitenames=2,
%   doi=false,
%   url=true,
%   isbn=false,
%   dashed=false
% ]{biblatex}
% \addbibresource{example.bib}

% example bibliography file
\usepackage{filecontents}
\begin{filecontents}{example.bib}
@book{example_book,
  author    = {Author, Anonymous},
  year      = {2000},
  title     = {The Definitive Resource},
  publisher = {Universal Press}
}
\end{filecontents}

\title{Documentation for the \texttt{automl} Package}

% The syntax for adding an author is
%
% \author[i]{\nameemail{author name}{author email}}
%
% where i is an affiliation counter. Authors may have
% multiple affiliations; e.g.:
%
% \author[1,2]{\nameemail{Anonymous}{anonymous@example.com}}

\author[1]{\nameemail{Author 1}{email1@example.com}}
\author[2,3]{\nameemail{Author 2}{email2@example.com}}
\author[3]{\nameemail{Author 3}{email3@example.com}}
\author[4]{\nameemail{Author 4}{email4@example.com}}

% if you need to force a linebreak in the author list, prepend an \author entry
% with \\:

\author[3]{\\\nameemail{Author 5}{email5@example.com}}

% Specify corresponding affiliations after authors, referring to counter used in
% \author:

\affil[1]{Institution 1}
\affil[2]{Institution 2}
\affil[3]{Institution 3}
\affil[4]{Institution 4}

% define PDF metadata, aids in accessibility of the resulting PDF
\hypersetup{%
  pdfauthor={AutoML}, % will be reset to "Anonymous" unless the "final" package option is given
  pdftitle={Documentation for the automl Package},
  pdfsubject={Documentation for automl Package},
  pdfkeywords={AutoML, LaTeX, style}
}

\begin{document}

\maketitle

\begin{abstract}
  Create with \verb|\begin{abstract} ... \end{abstract}|.
\end{abstract}

The \texttt{automl} package provides a \LaTeX{} style for the AutoML conference.
This document provides some notes regarding the package and tips for typesetting
manuscripts. The package and this document is maintained at the following GitHub
repository:
\begin{center}
  \url{https://github.com/automl-conf/LatexTemplate}
\end{center}
Users are encouraged to submit issues, bug reports, etc.\ to:
\begin{center}
  \url{https://github.com/automl-conf/LatexTemplate/issues}
\end{center}

A barebones submission is also available as
\texttt{barebones\_submission\_template.tex} in the same repository.

\section{Package Options}

With no options, the \texttt{automl} package prepares an anonymized manuscript
with hidden supplemental material. Two options are supported changing this
behavior:
\begin{itemize}
\item \texttt{final} -- produces non-anonymized camera-ready version for
  distribution and/or publication in the main conference track
\item \texttt{finalworkshop} -- produces non-anonymized camera-ready version for
  distribution and/or publication in the workshop track
\item \texttt{hidesupplement} -- hides supplementary material (following
  \verb|\appendix|); for example, for submitting or distributing the main paper
  without supplement
\end{itemize}
Note that \texttt{final} or \texttt{finalworkshop} may be used in combination
with \texttt{hidesupplement} to prepare a non-anonymized version of the main
paper with hidden supplement.

\section{Supplemental Material}

Please provide supplemental material in the main document. You may begin the
supplemental material using \verb|\appendix|. Any content following this command
will be suppressed in the final output if the \texttt{hidesupplement} option is
given.

\section{Note Regarding Line Numbering at Submission Time}

To ensure that line numbering works correctly with display math mode, please do
\emph{not} use \TeX{} primitives such as \verb|$$| and \texttt{eqnarray}.  (Using
these is not good practice anyway.)%
%
\footnote{\url{https://tex.stackexchange.com/questions/196/eqnarray-vs-align}}%
\footnote{\url{https://tex.stackexchange.com/questions/503/why-is-preferable-to}}
%
Please use \LaTeX{} equivalents such as \verb|\[ ... \]| (or
\verb|\begin{equation} ... \end{equation}|) and the \texttt{align} environment
from the \texttt{amsmath} package.%
%
\footnote{\url{http://tug.ctan.org/info/short-math-guide/short-math-guide.pdf}}

\section{References}

Authors may use any citation style as long as it is consistent throughout the
document. By default we propose author--year citations. Code is provided in the
preamble to achieve such citations using either \texttt{natbib/bibtex} or the
more modern \texttt{biblatex/biber}.

You may create a parenthetical reference with \verb|\citep|, such as appears at
the end of this sentence \citep{example_book}.  You may create a textual
reference using \verb|\citet|, as \citet{example_book} also demonstrated.

\section{Tables}

We recommend the \texttt{booktabs} package for creating tables, as demonstrated
in Table \ref{example_table}. Note that table captions appear \emph{above} tables.

\begin{table}
  \caption{An example table using the \texttt{booktabs} package.}
  \label{example_table}
  \centering
  \begin{tabular}{lrr}
    \toprule
    & \multicolumn{2}{c}{metric} \\
    \cmidrule{2-3}
    method & accuracy & time \\
    \midrule
    baseline & 10 & 100 \\
    our method & \textbf{100} & \textbf{10} \\
    \bottomrule
  \end{tabular}
\end{table}

\section{Figures and Subfigures}

The \texttt{automl} style loads the \texttt{subcaption} package, which may be
used to create and caption subfigures. Please note that this is
\emph{incompatible} with the (obsolete and deprecated) \texttt{subfigure}
package. A figure with subfigures is demonstrated in Figure
\ref{example_figure}. Note that figure captions appear \emph{below} figures.

Please ensure that all text appearing in figures (axis labels, legends, etc.)
is legible.

\begin{figure}
  \begin{subfigure}[t]{0.5\linewidth}
    \centering
    \framebox{Amazing figure!}
    \caption{Subfigure caption.}
    \label{example_figure_left}
  \end{subfigure}
  \begin{subfigure}[t]{0.5\linewidth}
    \centering
    \framebox{Another amazing figure!}
    \caption{Another subfigure caption.}
    \label{example_figure_right}
  \end{subfigure}
  \caption{An example figure with subfigures. \subref{example_figure_left}: an
    amazing figure. \subref{example_figure_right}: another amazing figure.}
  \label{example_figure}
\end{figure}

\section{Pseudocode}
\label{sec:code}

To add pseudocode, you may make use of any package you see fit -- the
\texttt{automl} package should be compatible with any of them. In particular,
you may want to check out the \texttt{algorithm2e}%
%
\footnote{\url{https://ctan.org/pkg/algorithm2e}}
%
and/or the \texttt{algorithmicx}%
%
\footnote{\url{https://ctan.org/pkg/algorithmicx}}
%
packages, both of which can produce nicely typeset pseudocode. You may also wish
to load the \texttt{algorithm}%
%
\footnote{\url{https://ctan.org/pkg/algorithms}}
%
package, which creates an \texttt{algorithm} floating environment you can access
with \verb|\begin{algorithm} ... \end{algorithm}|. This environment supports
\verb|\caption{}|, \verb|\label{}| and \verb|\ref{}|, etc.

\section{Adding Acknowledgments}

You may add acknowledgments of funding, etc.\ using the \texttt{acknowledgments}
environment. Acknowledgments will be automatically commented out at submission
time. An example is given below in the source code for this document; it will be
hidden in the \textsc{pdf} unless the \texttt{final} or \texttt{finalworkshop}
option is given.

\section{Required Material}

All submissions must include a discussion on limitations and a broader impact
statement; and a \textbf{Reproducibility Checklist} in their manuscripts, both
at submission and camera-ready time.
The discussion of limitations and broader impact is part of the 9 pages allocated
to the main paper (there is no page limitation for references and appendices), while
the reproducibility checklist is not.

\section{Broader Impact Statement}

The 9 pages allocated for the main paper must include a broader impact statement
regarding the approach, datasets and applications proposed/used in your
paper. It should reflect on the environmental, ethical and societal implications
of your work. The statement should require at most one page and must be included
both at submission and camera-ready time.

If authors have reflected on their work and determined that there are no likely
negative broader impacts, they may use the following statement:

\begin{displayquote}
After careful reflection, the authors have determined that this work presents no notable
negative impacts to society or the environment.
\end{displayquote}

This section is included in the template as a default, but you can also place these
discussions anywhere else in the main paper, e.g., in the introduction/future work.

The Centre for the Governance of AI has written an excellent guide for writing
good broader impact statements (for the NeurIPS conference) that may be a useful
resource for AutoML authors.%
%
\footnote{\url{https://medium.com/@GovAI/a-guide-to-writing-the-neurips-impact-statement-4293b723f832}}

\section{Submission Checklist}

All authors must include a section with the AutoML \textbf{Submission
Checklist} in their manuscripts, both at submission and camera-ready time.
The submission checklist is a combination of the
%
\href{https://neurips.cc/Conferences/2022/PaperInformation/PaperChecklist}
     {NeurIPS '22 checklist}
%
and the
\href{https://www.automl.org/wp-content/uploads/NAS/NAS_checklist.pdf}
     {\textsc{nas} checklist}.
%
For each question, change the default \verb|\answerTODO{}| (typeset \answerTODO)
to
\verb|\answerYes{[justification]}| (typeset \answerYes),
\verb|\answerNo{[justification]}| (typeset \answerNo), or
\verb|\answerNA{[justification]}| (typeset \answerNA).
\textbf{You must include a brief justification to your answer,} either by
referencing the appropriate section of your paper or providing a brief inline
description.  For example:
\begin{itemize}
\item Did you include the license of the code and datasets? \answerYes{See
  Section~\ref{sec:code}.}
\item Did you include all the code for running experiments? \answerNo{We include
  the code we wrote, but it depends on proprietary libraries for executing on a
  compute cluster and as such will not be runnable without modifications. We
  also include a runnable sequential version of the code that we also report
  experiments in the paper with.}
\item Did you include the license of the datasets? \answerNA{Our experiments
  were conducted on publicly available datasets and we have not introduced new
  datasets.}
\end{itemize}
Please note that if you answer a question with \verb|\answerNo{}|, we expect
that you compensate for it (e.g., if you cannot provide the full evaluation
code, you should at least provide code for a minimal reproduction of the main
insights of your paper).

Please do not modify the questions and only use the provided macros for your
answers. Note that this section does not count towards the page limit. In your
paper, please delete this instructions block and only keep the Checklist section
heading above along with the questions/answers below.
\begin{enumerate}
\item For all authors\dots
  %
  \begin{enumerate}
  \item Do the main claims made in the abstract and introduction accurately
    reflect the paper's contributions and scope?
    %
    \answerTODO{}
    %
  \item Did you describe the limitations of your work?
    %
    \answerTODO{}
    %
  \item Did you discuss any potential negative societal impacts of your work?
    %
    \answerTODO{}
    %
  \item Have you read the ethics review guidelines and ensured that your paper
    conforms to them? \url{https://automl.cc/ethics-accessibility/}
    %
    \answerTODO{}
    %
  \end{enumerate}
  %
\item If you are including theoretical results\dots
  %
  \begin{enumerate}
  \item Did you state the full set of assumptions of all theoretical results?
    %
    \answerTODO{}
    %
  \item Did you include complete proofs of all theoretical results?
    %
    \answerTODO{}
    %
  \end{enumerate}
  %
\item If you ran experiments\dots
  %
  \begin{enumerate}
  \item Did you include the code, data, and instructions needed to reproduce the
    main experimental results, including all requirements (e.g.,
    \texttt{requirements.txt} with explicit version), an instructive
    \texttt{README} with installation, and execution commands (either in the
    supplemental material or as a \textsc{url})?
    %
    \answerTODO{}
    %
  \item Did you include the raw results of running the given instructions on the
    given code and data?
    %
    \answerTODO{}
    %
  \item Did you include scripts and commands that can be used to generate the
    figures and tables in your paper based on the raw results of the code, data,
    and instructions given?
    %
    \answerTODO{}
    %
  \item Did you ensure sufficient code quality such that your code can be safely
    executed and the code is properly documented?
    %
    \answerTODO{}
    %
  \item Did you specify all the training details (e.g., data splits,
    pre-processing, search spaces, fixed hyperparameter settings, and how they
    were chosen)?
    %
    \answerTODO{}
    %
  \item Did you ensure that you compared different methods (including your own)
    exactly on the same benchmarks, including the same datasets, search space,
    code for training and hyperparameters for that code?
    %
    \answerTODO{}
    %
  \item Did you run ablation studies to assess the impact of different
    components of your approach?
    %
    \answerTODO{}
    %
  \item Did you use the same evaluation protocol for the methods being compared?
    %
    \answerTODO{}
    %
  \item Did you compare performance over time?
    %
    \answerTODO{}
    %
  \item Did you perform multiple runs of your experiments and report random seeds?
    %
    \answerTODO{}
    %
  \item Did you report error bars (e.g., with respect to the random seed after
    running experiments multiple times)?
    %
    \answerTODO{}
    %
  \item Did you use tabular or surrogate benchmarks for in-depth evaluations?
    %
    \answerTODO{}
    %
  \item Did you include the total amount of compute and the type of resources
    used (e.g., type of \textsc{gpu}s, internal cluster, or cloud provider)?
    %
    \answerTODO{}
    %
  \item Did you report how you tuned hyperparameters, and what time and
    resources this required (if they were not automatically tuned by your AutoML
    method, e.g. in a \textsc{nas} approach; and also hyperparameters of your
    own method)?
    %
    \answerTODO{}
    %
  \end{enumerate}
  %
\item If you are using existing assets (e.g., code, data, models) or
  curating/releasing new assets\dots
  %
  \begin{enumerate}
  \item If your work uses existing assets, did you cite the creators?
    %
    \answerTODO{}
    %
  \item Did you mention the license of the assets?
    %
    \answerTODO{}
    %
  \item Did you include any new assets either in the supplemental material or as
    a \textsc{url}?
    %
    \answerTODO{}
    %
  \item Did you discuss whether and how consent was obtained from people whose
    data you're using/curating?
    %
    \answerTODO{}
    %
  \item Did you discuss whether the data you are using/curating contains
    personally identifiable information or offensive content?
    %
    \answerTODO{}
    %
  \end{enumerate}
  %
\item If you used crowdsourcing or conducted research with human subjects\dots
  %
  \begin{enumerate}
  \item Did you include the full text of instructions given to participants and
    screenshots, if applicable?
    %
    \answerTODO{}
    %
  \item Did you describe any potential participant risks, with links to
    Institutional Review Board (\textsc{irb}) approvals, if applicable?
    %
    \answerTODO{}
    %
  \item Did you include the estimated hourly wage paid to participants and the
    total amount spent on participant compensation?
    %
    \answerTODO{}
    %
  \end{enumerate}
\end{enumerate}

\begin{acknowledgements}
  The authors have many people to thank!
\end{acknowledgements}

% print bibliography -- for bibtex / natbib, use:

\bibliography{example}

% and for biber / biblatex, use:

% \printbibliography

% supplemental material -- everything hereafter will be suppressed during
% submission time if the hidesupplement option is provided!
\appendix

\section{Proof of Theorem 1}

This material will be hidden if \texttt{hidesupplement} is provided.

\end{document}
